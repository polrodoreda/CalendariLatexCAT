%%%%%%%%%%%%%%%%%%%%%%%%%%%%%%%%%%%%%%%%%
% Calendari Mensual
% Plantilla LaTeX
%
% Autor original de l'estil:
% Evan Sultanik
%
% Nota important:
% Aquesta plantilla requereix l'arxiu calendari.sty al amteix directori que l'arxiu
% .tex. L'arxiu calendari.sty proveeix l'estructura necessària per crear el calendari.
%
%%%%%%%%%%%%%%%%%%%%%%%%%%%%%%%%%%%%%%%%%

%----------------------------------------------------------------------------------------
%	PAQUETS I COFIGURACIÓ DEL DOCUMENT
%----------------------------------------------------------------------------------------

\documentclass[landscape, a4paper]{article}

\usepackage{calendari} % Ús de l'estil calendar.sty

\usepackage[landscape, margin=0.5in]{geometry}

\begin{document}

\pagestyle{empty} % Elimina el número de pàgina al peu de pàgina

\noindent

\StartingDayNumber=1 % Dia d'inici del calendari, per defecte 1 significa dilluns, 2 per dimarts...

%----------------------------------------------------------------------------------------
%	SELECCIÓ DE MES I ANY
%----------------------------------------------------------------------------------------

\begin{center}
\textsc{\LARGE Mes}\\ % Mes
\textsc{\large Any} % Any
\end{center}

%----------------------------------------------------------------------------------------

\begin{calendar}{\hsize}

%----------------------------------------------------------------------------------------
%	DIES EN BLANC ABANS DEL COMENÇAMENT DEL MES
%----------------------------------------------------------------------------------------

% Aquesta part és molt delicada. Defineix el número de dies en blanc al començament del calendari abans de començar el mes. Si necessites que siguin més de 4, tens dues opcions:
% 1) Pots descomentar un o dos \BlankDay següents que faran una nova setmana (6 en total) que farà el calendari massa gran per una sola pàgina; el remei passa per decrementar la mida de cada dia substituïnt els 2.5cm següents per un nombre menor.
% 2) Fes que els dies que "sobren" comencin al principi esquerra del calendari (el calendari comença amb 31, després uns dies en blanc, després 1, 2, etc). La segona opció pot ser aplicada descomentant el següent:

%\setcounter{calendardate}{31} % Comença el compte amb 31, per tant el primer dia del mes serà 31; aquest valor pot ser modificat a 30 o 29 segons convingui.
%\day{}{\vspace{2.5cm}} % 31 - afegeix una nova línia idèntica si comença al 30 o després

% Serà necessari que comentis el 31 a la secció DIES ENUMERATS I CONTINGUT DEL CALENDARI.

\BlankDay
\BlankDay
\BlankDay
%\BlankDay
%\BlankDay
%\BlankDay

%----------------------------------------------------------------------------------------
%	DIES ENUMERATS I CONTINGUT DEL CALENDARI
%----------------------------------------------------------------------------------------

% Aquí es defineixen els dies enumerats a la plantilla; si n'hi ha menys de 31, simplement s'han de comentar les últimes línies.

% \vspace{2.5cm} Especifica la mida del dia

\setcounter{calendardate}{1} % Inicia la conta a 1

%\day{Universitat}{10:00 Examen \\[6pt] 12:00 Presentació} % 1 - Exemple
\day{}{\vspace{2.5cm}} % 1
\day{}{\vspace{2.5cm}} % 2
\day{}{\vspace{2.5cm}} % 3
\day{}{\vspace{2.5cm}} % 4
\day{}{\vspace{2.5cm}} % 5
\day{}{\vspace{2.5cm}} % 6
\day{}{\vspace{2.5cm}} % 7
\day{}{\vspace{2.5cm}} % 8
\day{}{\vspace{2.5cm}} % 9
\day{}{\vspace{2.5cm}} % 10
\day{}{\vspace{2.5cm}} % 11
\day{}{\vspace{2.5cm}} % 12
\day{}{\vspace{2.5cm}} % 13
\day{}{\vspace{2.5cm}} % 14
\day{}{\vspace{2.5cm}} % 15
\day{}{\vspace{2.5cm}} % 16
\day{}{\vspace{2.5cm}} % 17
\day{}{\vspace{2.5cm}} % 18
\day{}{\vspace{2.5cm}} % 19
\day{}{\vspace{2.5cm}} % 20
\day{}{\vspace{2.5cm}} % 21
\day{}{\vspace{2.5cm}} % 22
\day{}{\vspace{2.5cm}} % 23
\day{}{\vspace{2.5cm}} % 24
\day{}{\vspace{2.5cm}} % 25
\day{}{\vspace{2.5cm}} % 26
\day{}{\vspace{2.5cm}} % 27
\day{}{\vspace{2.5cm}} % 28
\day{}{\vspace{2.5cm}} % 29
\day{}{\vspace{2.5cm}} % 30
\day{}{\vspace{2.5cm}} % 31

% Descomenta el \BlankDay següent si falta la línia de sota el calendari
%\BlankDay

% Descomenta per tornar a començar després del 31
%\setcounter{calendardate}{1}
%\day{}{\vspace{2.5cm}} % 1
%\day{}{\vspace{2.5cm}} % 2
%\day{}{\vspace{2.5cm}} % 3

%----------------------------------------------------------------------------------------

\finishCalendar
\end{calendar}
\end{document}
